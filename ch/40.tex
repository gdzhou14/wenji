\vspace{10pt}

{\centering\subsection*{梁好:会飞的木屋}}

\addcontentsline{toc}{subsection}{梁好:会飞的木屋}

\renewcommand{\leftmark}{梁好:会飞的木屋}

\begin{figure}[htbp]

\centering

\includegraphics[width = .5\textwidth]{./ch/40.jpg}

\end{figure}



我现在非常非常地想要一个东西,我相信你们也一定非常非常地想要这个东西,它就是——会飞的木屋。这个木屋的底下有四个轮子,两侧有能缩能伸的机翼,上面有一根“触角”,那是全息红外线热能扫描仪。它可以随意变换颜色,它有五头大象那么高。最重要的是,它有非常多的功能:一、能遨游太空。到了太空后,它能无死角封住,不让氧气泄露出去,同时还会打开氧气舱放出氧气。它可以启动重力模式,不让东西飘起来。如果想体验失重的感觉时,关掉重力模式就可以了。二、自动除尘功能。可以让木屋一直保持干净。如果你鞋子上有淤泥之类的,它就会自动去尘。这样子,就可以不用打扫屋子了。三、变大变小。有了变大变小的功能,就可以随身携带。是不是很像孙悟空的法宝金箍棒呀。四、自主修复。有了自主修复功能,就可以在第一时间修复并保护使用者的安全。使用方法是这样的:有一台遥控器,按红色的按钮并说出想到的地方,就能到达;按绿色的按钮会自动打扫;按黑色的按钮可变大变小......如果你记不住的话,没关系,在遥控器的按钮旁都标注有它的作用。如果不识字的话,也没有关系,只要按下白色按钮并说出你想要启动什么样的按钮,那个对应的按钮就会发出红色的亮光。对了,差点忘了说发明它的原因了。因为我想要遨游世界想让妈妈不再那么辛苦......这就是我想要发明的东西,你们有哪些奇思妙想呢?

点评:既能看到想象力(或是功能需求描述力),又能看到现实生活物品和场景的"影子"。惊讶于“全息红外线热能扫描仪”和“重力模式”这样的高级词汇,感动于发明的理由。列举出功能并解释原因,不错的一种写作方式。





\vspace{10pt}



作者:四(1)班梁好

指导老师:周瑞原标题:《我的奇思妙想》

投稿:2021年4月15日

发表:2021年4月16日


                



\vspace{10pt}

\hline



