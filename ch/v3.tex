\vspace{10pt}

{\centering\subsection*{龙柏泉:如何营造良好的家庭阅读环境?}}

\addcontentsline{toc}{subsection}{龙柏泉:如何营造良好的家庭阅读环境?}

\renewcommand{\leftmark}{龙柏泉:如何营造良好的家庭阅读环境?}

\begin{figure}[htbp]

\centering

\includegraphics[width = .5\textwidth]{./ch/v3.jpg}

\end{figure}

大家好!在上一周,何贤文的妈妈分享了家庭读书的视频,这是很好的家庭氛围。良好的家庭阅读环境非常重要,我们想对这个话题做进一步的探讨。

 

首先,我想表达一下我对于阅读的认识。

 

我认为,阅读就是一个聆听与思考的过程。作家一辈子的经历与感悟,往往就浓缩在他们写的几本书中。多读几位作家的书,某种程度上来说,就是体验了几种不同的人生。那么阅读就是一个不断拓宽人生边界的过程。在自己可以支配的时间里多多阅读,是非常值得的一件事。

 

俗话说,兴趣是最好的老师。爱读书的人往往对于这个世界有强烈的好奇心。他们想知道这个世界从前发生过的事情,想象未来可能发生的事情。他们在阅读的过程中,不断探寻事物的演化过程,挖掘人性的深刻内涵,总结历史的经验教训。在这个不断求索的过程中,他们获得了精神上的快乐。潜移默化之中,他们往往比不爱读书的人更有温度,更懂情趣,更加具有独立思考的能力。

 

回到今天的话题——对于小学阶段的读者,如何为他们创造一个良好的家庭阅读环境呢?

 

第一点,要能接触到书。家长们可以到图书馆借阅,或是购买一些适合小学生读的书。比如,童话故事、历史人物、世界地理、动物植物、少年文学、科学幻想等等。这些类型的书浅显易懂,生动有趣。既可以增长知识,又可以激发他们的阅读欲望。假如父母本身工作忙,或者根本不在身边,那么可以试试在家里各个角落都随意放上几本书,增加孩子打开书翻看书的机会。

 

第二点,要固定安排阅读时间。贪玩和懒惰是人的天性,何况还是懵懵懂懂的小学生。对于不爱主动阅读的孩子,家长可以在他们睡觉之前,安排半个小时的阅读时间。如果家长精力允许,可以陪在他们身边一起阅读,这也是一件其乐融融的事情。

 

第三点,要有安静的环境。小孩子思维活跃,容易被干扰和诱惑。在他们的阅读时间里,尽量不要大声说话,也不要看电视,让孩子可以静下心来阅读。

 

第四点,适当激励。通过奖励启动阅读,培养习惯,克服开始阅读的困难和障碍。比如,跟孩子约定,读完一本书,给点零花钱,或者完成他的一个小愿望。在奖赏的激励下,孩子阅读的热情会提高不少。

 

第五点,尊重孩子的选择。假如孩子实在是不喜欢阅读,那么家长也不要勉强,以免产生抵触情绪,适得其反。人生的道路那么长,在各种因缘际会之下,他们后来爱上阅读也说不定。

 

以上是我个人的一些想法!谢谢大家!


\vspace{10pt}


作者:龙柏泉

朗读:龙柏泉

发布:2021年4月24日








\vspace{10pt}

\hline

