\section{附录}
\vspace{10pt}

{\centering\subsection*{桂花图书馆简介}}

\addcontentsline{toc}{subsection}{桂花图书馆简介}

\renewcommand{\leftmark}{桂花图书馆简介}


   愿景:让阅读成为乡村孩子的基本福利!
   
   使命:陪伴乡村孩子的精神成长!
   
    桂花图书馆是湖北省咸宁市首家乡村公益图书馆,位于咸宁市咸安区桂花镇柏墩中心小学。由原校长顾怀强老师倡议,并由小学校友发起。桂花图书馆旨以图书为载体,在乡村地区推广阅读,陪伴乡村孩子的精神成长,帮助乡村孩子开拓视野。
    
    图书馆于2014年5月5日正式成立,刘道玉教育基金会秘书长何莹和湖北新民人文教育研究所代表冯宜山于当日莅临了开馆仪式。截止2020年12月份,经统计汇总,有藏书6950册,主要为儿童文学、科普、国学、人物传记、教育心理学类书籍,图书馆服务小学将近1000名师生。
    
    第四届(2020-2022)执行团队成员(按拼音排序):冯静(理事)、顾亚男(财长)、龙柏泉(理事)、莫静琴(馆长)、叶成(监事长)、张锦(理事长)、周国栋(理事)。
    

    2014年5月-6月,由张国庆老师担任馆长负责图书馆运营,尝试借阅模式。
    
    2014年9月-2015年6月,由刘蕾老师担任馆长。三到六年级以班级为单位借阅,每两周更换图书一次。
    
    2015年4月-6月,长期志愿者刘艺全职管理图书馆,三至六年级开始可以个人借阅。
    
    2015年9月-2016年6月,由张锦担任桂花图书馆全职馆长。
    
    2016年9月-2017年6月,由刘娟老师担任桂花图书馆全职馆长。
    
    2017年9月-2018年6月,由颜伟老师担任桂花图书馆全职馆长。
    
    2018年9月-2019年12月,由张秀荣老师担任桂花图书馆全职馆长。
    
    2020年上半年,由于新冠疫情学校关闭,未开馆。
    
    2020年9月起,由莫静琴老师担任桂花图书馆全职馆长。其中12月完成书籍的电子录入,建立数据库和电子借阅平台,12月中旬开始电子借阅。2021年10月起,因双减政策,图书馆交于学校管理。
    
    
    



\vspace{10pt}

\hline



\vspace{10pt}

{\centering\subsection*{法律声明}}

\addcontentsline{toc}{subsection}{法律声明}

\renewcommand{\leftmark}{法律声明}

《桂花图书馆文集(2021)》为桂花图书馆邀请的序言作者、读者、小义工、小编辑、志愿者、桂花之友和执行团队在2021年发表的文章的汇集。文集为非卖品,仅供免费分享,但桂花图书馆保留文集的版权,任何组织和个人不得以任何方式以此文集营利。


文集采用的图片,大部分来自于互联网。文集的文章均在桂花图书馆微信公众号“桂花之友”中发表,发表的文章中有标注引用来源。文集图片采用如有侵权,欢迎图片作者通过邮箱guihualib@qq.com与桂花图书馆联系,谢谢。


特此声明。
\vspace{10pt}

桂花图书馆


2022年1月


\vspace{10pt}

\hline



\clearpage